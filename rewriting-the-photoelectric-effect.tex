\documentclass{article}
\usepackage{amsmath, amssymb}
\usepackage{graphicx}

\title{Rewriting the Photoelectric Effect Using Recursive Radiation Transfer Instead of Particle-Wave Duality}
\author{}
\date{}

\begin{document}

\maketitle

\section{Introduction}
The photoelectric effect has historically been cited as proof of the existence of photons, as light appears to eject electrons only when reaching a threshold frequency. The standard model assumes that light behaves as discrete particles (quanta) rather than continuous waves. However, this assumption arises from a fundamental misunderstanding of how radiation interacts with mass structures. Instead of invoking photons, we introduce the concept of \textbf{Recursive Radiation Transfer}, which accounts for energy exchange through recursive Surface Interactions.

\section{Flaws in the Photon-Based Explanation}
Einstein's photon hypothesis states:
\begin{equation}
    E = h f
\end{equation}
where $E$ is the energy, $h$ is Planck’s constant, and $f$ is the frequency of incident light. The conventional interpretation is:
\begin{itemize}
    \item Electrons absorb an entire photon or nothing at all.
    \item If light is below a threshold frequency, no electrons are ejected, regardless of intensity.
    \item If the frequency is high enough, electrons are ejected with kinetic energy proportional to frequency.
\end{itemize}
This is taken as evidence that light must be a particle because classical wave-based energy accumulation should allow gradual energy transfer, but this is not observed.

\section{Recursive Radiation Transfer Model (RRTM)}
Instead of treating light as discrete particles, we propose that \textbf{radiation is a recursive energy redistribution process} where:
\begin{itemize}
    \item Electrons do not absorb ``photons,'' but instead absorb radiation based on recursive Surface Interactions.
    \item Energy absorption is determined by resonance conditions between the Radiation Source and the absorbing Mass Structure.
    \item The threshold frequency is a function of resonance, not quantization.
\end{itemize}

\subsection{Corrected Energy Absorption Equation}
Instead of Equation (1), we use:
\begin{equation}
    E_{\text{absorbed}} = \left( \frac{G M}{r^2} \cdot f(n) \right) \cdot \alpha_{\text{surface}}
\end{equation}
where:
\begin{itemize}
    \item $G$ is the gravitational constant, linking radiation transfer to stored gravitation.
    \item $M$ is the mass of the absorbing structure, meaning energy absorption depends on mass properties.
    \item $r$ is the distance from the radiation source, showing that Surface Interactions depend on scale.
    \item $f(n)$ is the recursive function of the Degree of Surface Interaction, meaning energy transfer follows a fractal-like structure.
    \item $\alpha_{\text{surface}}$ is the absorption probability, dependent on material structure and resonance conditions.
\end{itemize}

\section{Explaining Observed Effects}
\subsection{Threshold Frequency}
The so-called ``threshold frequency'' is a resonance condition between the extended radiation and the material's Degrees of Surface Interaction. Below this frequency, energy is not efficiently stored in the electron’s structure, so no ejection occurs.

\subsection{Instantaneous Ejection of Electrons}
Energy transfer appears instantaneous because the correct resonance conditions allow efficient energy absorption, not because of discrete photons.

\subsection{Intensity Below Threshold Does Nothing}
If the incoming radiation does not match the correct Surface Interaction conditions, increasing intensity does not help, explaining why low-frequency light cannot eject electrons regardless of intensity.

\section{Implications and Next Steps}
This model eliminates the need for photons while preserving all experimental observations. Instead of wave-particle duality, we recognize that:
\begin{itemize}
    \item Light is not a particle or a wave but a recursive radiation transfer process.
    \item The observed quantization in the photoelectric effect arises from resonance conditions, not photon absorption.
    \item All energy interactions are governed by Degrees of Surface Interaction, meaning energy absorption is not a binary event but a recursive process.
\end{itemize}

Future work could apply this to:
\begin{itemize}
    \item Explaining Compton scattering without photons.
    \item Correcting blackbody radiation models.
    \item Demonstrating why quantum mechanics misinterpreted Planck’s constant as proof of discrete energy packets instead of resonance.
\end{itemize}

\end{document}
